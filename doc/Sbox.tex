\documentclass{xduugthesis}
% \usepackage{amsmath}
% \usepackage{amsfonts}
\usepackage{algorithm}
\usepackage{algorithmic}
\usepackage{graphicx}
\graphicspath{{/Users/grhunhun/Desktop/xdutex/pic/}}
% 不要写这些
% \usepackage{amsmath}
% \usepackage{amsfonts}
% \usepackage{biblatex}
% 添加参考文献数据库
% \addbibresource{demo.bib}
% \numberwithin{equation}{section}
%---------------开始----------------
% set up
\xdusetup{
    info = {
        title = {S盒的密码学性质检测},
        department = {网络与信息安全学院},
        major = {信息安全专业},
        author = {龚若涵},
        supervisor = {王子龙},
        % supervisor-department = {xdu},
        class-id = {1818012},
        student-id = {18180100149},
        abstract = {abstract-zh.tex},
        abstract* = {abstract-en.tex},
        keywords = {S盒,布尔函数,设计准则,密码学},
        keywords* = {S-box,Boolean function,Design criteria,Cryptography},
        acknowledgements = {acknowledgements.tex}
},
    style = {
        symmetric-margin = true,
        before-skip = {24pt, 20pt, 12pt, 12pt, 12pt, 12pt},
        after-skip = {24pt, 18pt, 12pt, 12pt, 12pt, 12pt},
        font-type = file,
        font-path = {/Users/grhunhun/Desktop/xdutex/XDUTeXfont},
        cjk-font = adobe,
        latin-font = tacn,
        math-font = xits,
        % algorithm-small-font = false, % 设置算法内容字体是否位五号
        bib-resource = {/Users/grhunhun/Desktop/xdutex/demo.bib},
    }
}
% \theoremstyle{definition}
\newtheorem{definition}{定义}
\newtheorem{mytheorem}{定理}[section]
\renewcommand{\algorithmicrequire}{\textbf{Input:}}
\renewcommand{\algorithmicensure}{\textbf{Output:}}


%! 导言区结束
\begin{document}
\frontmatter
\mainmatter
\chapter{引言}
% 引言部分包括问题的提出及其背景、国内外现状、前人所作的结果等
\section{研究背景及其意义}
%! 测试
参考文献测试\cite{Magma},这是我的参考文献


分组密码作为密码学中的一个重要分支,具有加解密速度快、易实现等优点。而作为许多分组算法中的非线性部件,S 盒(Substitution Box, S-box) 起着支撑整个算法安全性的作用。因此,S盒的密码强度决定了整个分组密码算法的安全强度。但如何全面准确地度量它的密码学强度,用更简单通用的方法检测其各个密码学性质,一直以来是密码设计与分析的研究难题。\par
S盒首次出现在Lucifer算法中,之后应用于很多的密码算法中,最典型的例子即为数据加密标准(Data Encryption Standard,DES)和高级加密标准(Advanced Encryption Standard,AES)。一般地,$n\times m$ 的S盒本质上可以看作一个映射:$F(x) = (f_1(x), f_2(x),\dots , f_m(x)), x\in \mathbb{F}_2^n$ ,相当于$m$个$ n$元布尔函数的线性组合。当$n$和$m$均很大时,几乎所有S盒都是非线性的,但这会带来存储和运算上的困难\cite{冯登国2000分组密码的设计与分析}。我们希望在较小的存储量下获得好的S盒,同时具备必要的安全性,因此可以通过计算一系列密码学性质,来检测该S盒抵抗某种攻击的强度,从而全面准确地度量S盒的密码强度。\par
在密码学研究中有两个重要分支:密码编码学和密码分析。其中,密码编码学是研究如何对信息进行编码以保证信息的认证性和保密性的学科,密码分析学是研究如何分析密码体系并加以攻击来获得信息的学科。为了更深入地研究分组密码,需要设计满足不同要求的密码函数。作为分组密码中设计的难点,如何设计密码性质较好的S盒被许多研究者所关注。对S盒进行密码学性质进行检测,可以直观地给出各项指标,有助于密码设计者快速找到满足某些特定密码需求的新的密码函数。这些指标往往主要来源于S盒的设计准则和构造方法。同时,对S盒的深入研究不仅有助于迭代分组密码的设计,而且对于以非线性变换为核心的密码算法的密码分析也有相当价值。检测的性质中有非线性度\cite{pieprzyk1988towards}、差分均匀度\cite{DBLP:conf/fse/Nyberg94}、雪崩效应、轮换对称性等,这些性质分别对应着抵抗不同攻击方法的强度,例如非线性度决定着对应密码算法抵抗线性分析攻击的能力,如果计算结果较差,说明该密码算法容易遭到差分攻击。这样可以给密码分析者提供攻击方向的参考。从整体来看,由于S盒的安全要求取决于所有密码部件的设计策略,每一种安全要求对应着抵抗不同的攻击,因此,设计者需要对抵抗各种攻击的安全性和效率之间进行权衡,以达到总体设计的优越。而S盒的性质检测可以为设计者或者分析者提供各个方面的参考,全面准确地度量S盒的密码学强度。\par
另一方面,基于工具的易用性考虑,当下亟需更简单通用的方法来检测S盒各个密码学性质。尽管已有平台例如SageMath\cite{SageMath},能够计算S盒的部分密码学性质,但评估并不全面,而且对于用户而言操作繁琐,使用困难。目前市面上仍然缺乏国产化、轻量级、评估全面的S盒分析检测工具。因此,设计一个简单通用的S盒密码学性质检测软件是十分必要的。\par

\section{国内外研究现状}
\subsection{S盒密码学性质研究现状}
在密码设计和分析中,S盒起着极为重要的作用,关于S盒的研究也从未停止。在密码学中应使用具有良好性质的 S 盒来抵御各种攻击,同时密码设计者也需要在安全性和效率之间进行权衡。因此重点研究S盒的密码学性质和已有的检测工具,其研究进展如下。\par

现有对称密码算法设计仍沿用香农1949年提出的混淆和扩散思想\cite{DBLP:journals/bstj/Shannon49a},是指通过具有混淆和扩散特性的密码部件使得明文、密文和密钥之间的关系非常复杂,以至于攻击者无法从密文中得到明文的任何信息或者从明文密文对中得到密钥的任何信息。许多分组加密算法都是基于S盒的密码强度,例如AES算法、韩国对称加密标准SEED算法、欧洲对称加密标准Camellia算法和中国商用密码标准SMS4算法等。因此,国内学者对于具体的密码算法中S盒分析的研究较多。例如,文献\cite{刘佳2013对称密码算法}就对上述四种密码算法进行了研究探讨,分别检测了代数性质和布尔函数性质,分析各种算法抵抗差分密码分析和线性密码分析等攻击的能力。另外,刘景伟等人针对 AES 中的 S 盒密码学性质进行了深入的研究,采用一种简单通用的拉格朗日插值法推导出 S 盒的代数表达式,计算并证明了 S 盒的差分均匀度、平衡性、严格雪崩准则、代数次数等性质\cite{刘景伟2004aes}。针对通用的 S 盒,刘晓晨等人基于 m-序列和正形置换分别构造了 8 比特和 4 比特的安全 S 盒 \cite{刘晓晨2000满足若干密码学性质的}。另外,研究还包括 S 盒的旋转对称性\cite{高光普2017旋转对称布尔函数研究综述},抗差分性与非线性 \cite{吴丹2011s}等。\par
一般地,对已知密码算法的各种攻击方法导致所使用的密码 S 盒必须满足一些密码学指标,所以研究者通常使用S盒的密码学性质来表示密码算法抵御攻击的能力。S 盒的传统的密码学性质包括:差分均匀度、线性度及相关免疫度等。1985年,Webster 和 Tavares 提出了 S 盒的两个密码学性质\cite{webster1985design}:完备性(completeness)和雪崩性(avalanche)。1991 年,Nyberg 针对差分攻击提出了完美非线性 S 盒理论 \cite{nyberg1991perfect}:在S盒的输入变量的长度至少为输出变量的两倍的情况下,不会受到差分密码分析的攻击。2007年,Leander等人对全体4 x 4的S 盒进行了分类 \cite{leander2007classification}。该分类结合线性与差分分析,通过一种 linear equivalence 的概念其划分成16类,并表明可以从这类S盒中研究S盒安全性。2020 年, Dey 等针对4元布尔函数的平衡性、线性度和严格雪崩准则等密码学性质进行研究,并对生成的S盒进行密码分析\cite{dey20204}。\par

可以看出,更多的对S盒的密码学性质研究分散在不同的方面,往往只局限于某一具体的密码体系或密码算法,或者只对抵抗某种具体攻击来进行分析。例如,文献\cite{杨斌2013s}主要研究了应用在序列密码中的S盒,对欧洲NESSIE计划和eSTREAM计划中所涉及到的利用分组密码部件S盒构造流密码的情况进行了分析。但是,现有研究中针对某一个密码学性质的专门研究非常丰富且范围广泛,将会在本文第三章进行介绍。\par
\subsection{现有检测工具调研}
研究S盒密码学性质的平台SageMath\cite{SageMath}是较为常见的一个开源数学工具。该工具包含了从线性代数、微积分,到密码学、群论、图论、数论等各种初高等数学的计算功能。而且SageMath内置了专门用于密码学计算的模块,其中sage.crypto模块可以用于评估S盒的许多重要密码学性质。例如,sage.crypto.Sbox模块可以对任意输入的S盒进行代数处理和性质评估,比如给出S盒的差分分布表(Differential Distribution Tables DDT),非线性度等;而sage.crypto.Sboxes模块提供了许多常用的密码算法中的S盒及其密码学性质。但是,SagaMath覆盖范围并不全面,暂时还不支持部分重要的密码学性质检测,例如对(v,w)线性度的检测。同时,在检测大量S盒的情况下,该工具的效率并不高,检测时间比较久。除了SageMath之外, Magma\cite{Magma}是一款由悉尼大学数学与统计学系计算代数学小组开发的功能强大的代数计算程序包,该软件专门解决代数系统中的数论、代数几何和代数组合学的计算问题,也包括密码学模块,对于研究S盒非常方便。\par
另一个GitHub的开源项目libapn\cite{libapn}主要用于研究布尔函数,包括但不限于APN函数。它可以用于计算DDT、差分均匀度、代数次数以及寻找APN函数。但是,libapn只考虑了有关抵抗差分攻击的安全属性,其他的性质并没有覆盖到。\par
另外,还有一些函数库也可以用于对S盒密码学性质的检测。例如,R是一个可以用于统计分析的数学编程语言。其中可加载的boolfun模块可以用来评估布尔函数的部分密码学属性,例如非线性度、免疫性等,同时也提供了处理布尔多项式的功能。VBF库(Vector Boolean Fuction Library)是由Alverez-Cubero 和 Zufiria 提出的从密码学角度进行布尔函数分析的工具,可用于计算S盒的各个密码学性质\cite{alvarez2016algorithm}。\par
文献\cite{DBLP:journals/tosc/BaoGLS19}中提出了一个名为PEIGEN的平台,可以用来评估S盒的安全强度,并给出高效的软硬件实现。该平台集成了大部分现有工具的功能特性,检测性质范围比较全面,也使用了效率更高的搜索算法,可以为S盒的研究与设计提供系统性的参考。不过该平台主要是对n-bit S盒(3≤n≤8)进行研究,对于更大的S盒(n≥5位),它仅用于评估安全性,但还不足以完成S盒的实现和生成。但是,该平台暂时没有可用的UI界面。类似地,Picek等人在2014年提出了一个简单的S盒分析工具\cite{DBLP:conf/wistp/PicekBJEG14}
,提供了检测布尔函数和S盒的主流密码学性质的功能。\par
随着密码学技术的不断发展,未来会有更多层出不穷的攻击出现。每当新的攻击方式出现,针对抵抗这些攻击的安全属性进行研究与检测是非常有必要的。例如,2011年针对轻量级分组密码算法PRINTcipher提出的invariant subspace攻击\cite{leander2011cryptanalysis},引起了研究者对于此方面的关注。而在此之前,此类密码学性质并没有被注意到,也缺少相关的研究和检测分析。因此,随着密码学技术的发展,密码分析技术的更新,设计一个安全的密码算法需要考虑的方面会愈加复杂。而一个通用、用户友好且评估全面的S盒的密码学检测工具将会为密码学的研究与设计提供系统性的参考和助力。

\section{论文结构安排}
论文一共分为五章,其中:\par
第一章首先介绍了本文的研究背景及其意义,并从两个角度:S盒密码学性质研究现状和现有检测工具调研,说明了国内外的发展现状的调研。\par
第二章主要介绍了预备知识,对布尔函数和S盒作必要的说明。

\chapter{预备知识}
\section{符号说明}
%! 符号表格
以下表格中列出了后文中使用的部分符号及其说明,完整的符号说明表格附在附录A。\par
\begin{tabular}{|p{3cm}|p{11cm}|}
\hline
符号& 说明\\
\hline
$\oplus,+,\sum $ & 文中使用的$\oplus$,$+$或者$\sum$都表示$\mathbb{F}_2^n$上的加法运算,整数域$\mathbb{Z}$上的加法运算会另加说明。\\
\hline
$a$ & a为二元有限域上的二进制向量,$a=(a_1,a_2,\dots ,a_n)$ ,其中$a_i \in \mathbb{F}_2$是$a$的下标为$i$的坐标(\emph{coordinate})。\\
\hline
$a \cdot b$ & 表示两个二进制向量$a,b \in \mathbb{F}_2^n$ 的内积,其中$a \cdot b \triangleq \oplus _{i=1}^n a_i \cdot b_i$。\\
\hline
$wt(a)$ & 二进制向量$a \in \mathbb{F}_2^n$ 的汉明重量(\emph{Hamming weight}),定义$wt(a)\triangleq \sum_{i=1}^n a_i $。\\
\hline
$supp(a)$ & 二进制向量$a\in \mathbb{F}_2^n$ 满足$f(a)=1$的全体元素集合称为$f$的支撑后者开集(\emph{support})。\\
\hline
\end{tabular}\par

\section{布尔函数}
%! 布尔函数

布尔函数(\emph{Boolean function})是许多密码系统的核心部件,其密码学性质的优劣决定着整个密码系统的安全性。同时,布尔函数的密码学性质与S盒联系非常密切。
因此,这一小节会介绍布尔函数的定义及其表示。\par
以$\mathbb{F}_2^n$表示所有$n$元组$(x_1, x_2, \dots ,x_n)$,$x_i\in \mathbb{F}_2$构成的集合,$\mathbb{F}_2$表示含有两个元素的有限域。
$f$是从$\mathbb{F}_2^n$到$\mathbb{F}_2$的映射,则称$f$是一个$n$元布尔函数,记作$f(x)$, $x\in \mathbb{F}_2^n$。
$$
    f: \mathbb{F}_2^n \mapsto \mathbb{F}_2
$$
通常布尔函数有三种常见的表示方式:\par
1)真值表表示。由于布尔函数的定义域和值域都是有限集,那么可以把函数的对应关系一一列举出来,这样的表示方法即为真值表表示法。
假设指定$\mathbb{F}_2^n$中元素的一个排列顺序,并将布尔函数$f$在这些元素上的取值按照顺序写入向量,
可以得到二元有限域上的一个向量,长度为$2^n$,该向量就称为布尔函数的$f$的真值表。
在本文中,默认将元素$x=(x_1, x_2, \dots , x_n)$中的$x_1$视作最低位,$x_n$视作最高位,并按照二进制表示的整数值递增排列,
因此布尔函数的真值表形如:
$$
    (f(0,\dots ,0,0), f(0,\dots ,0,1), f(0,\dots ,1,0),\dots ,f(1,\dots ,1,1))
$$\par

% \newtheorem{mytheorem}{定理}[section]
\begin{mytheorem}[布尔函数的平衡性]\label{thm:balance}
    若一个$n$元布尔函数$f$的值向量满足$wt(f)=2^{n-1}$,那么称该布尔函数是平衡(\emph{balanced})的,即
    \begin{equation}
        |\{x\in \mathbb{F}_2^n |f(x)=1 \}| = |\{x\in \mathbb{F}_2^n |f(x)=0 \}| = 2^{n-1} 
    \end{equation}\par
\end{mytheorem}\par

2)小项表示。数学上,我们还常常使用含n个变元的多项式来表示n元布尔函数:
\begin{equation}
    f(x_1,x_2,\dots ,x_n) = \mathop{\oplus}\limits_{a_i \in \mathbb{F}_2}f(a_1,a_2,\dots,a_n)x_1^{a_1}x_2^{a_2}\dots x_n^{a_n}
\end{equation}\par

3)代数正规型表示。
\begin{definition}[布尔函数的代数正规型]
n元布尔函数$f:\mathbb{F}_2^n \mapsto \mathbb{F}_2$可以如(\ref{def:BfAnf})的表示形式存在且唯一,称为$f$的代数正规型(ANF, algebraic normal form)。
\begin{equation}
f(x_1,x_2,\dots ,x_n) = \mathop{\oplus}\limits_{u \in F_2}\alpha_u \prod\limits_{i=1}^n x_i^{u_i}\mbox{,其中}\alpha_u \in \mathbb{F}_2 \label{def:BfAnf}
\end{equation}
\end{definition}\par

另外,布尔函数的真值表$f$或者ANF之间能按照如下公式\ref{anftov} 进行计算转换,其中$\alpha_u$可以视作一个由莫比乌斯反演变换todo定义的$\mathbb{F}_2^n$到$\mathbb{F}_2$的布尔函数。
\begin{equation}
\alpha_u =\mathop{\oplus}\limits_{x\preceq u}f(x), \quad f(x)=\mathop{\oplus}\limits_{u\preceq x} \label{anftov}
\end{equation}
其中,$x$满足关系$x\preceq u$当且仅当对于所有的$1 \le i \le n$,有$x_i \le u_i$。\par

\begin{definition}[布尔函数的代数次数]
    非零布尔函数$f:\mathbb{F}_2^n \mapsto \mathbb{F}_2$的代数正规型中系数非零项所含有的最多变元的个数称为$f$的代数次数,记作deg(f)。
\end{definition}\par
布尔函数的代数次数也等于其ANF中出现在乘积项中的$x_i$的最高次数。如果
$$
ANF_f =\mathop{\oplus}\limits_{u \in F_2}\alpha_u \prod\limits_{i=0}^{n-1} x_i^{u_i},\quad \alpha_u \in \mathbb{F}_2
$$\par
则有
\begin{equation}\label{bfAlgebraicDegree}
    deg(f) \triangleq max\{wt(u)|u\in \mathbb{F}_2^n,\alpha_u \ne 0 \in \mathbb{F}_2^m \enspace in \enspace ANF_f \}
\end{equation}\par

\begin{definition}[仿射函数、线性布尔函数]
    代数次数不超过1的布尔函数称为仿射函数(Affine functions),将全体n元仿射函数的集合记作$A_n$。为方便起见,后面用$\varphi_{\alpha}$表示某个仿射函数。
\begin{equation}\label{affine_func}
    \begin{aligned}
        A_n &\triangleq \{ f(x_1,x_2,\dots ,x_n) = \mathop{\oplus}\limits_{i=1}^n a_ix_i + a_0 |a_i \in F_2 \} \\
        &= \{\varphi_{\alpha}, \varphi_{\alpha} \oplus 1|\varphi_{\alpha}:x \mapsto \alpha \cdot x, \alpha \in \mathbb{F}_2^n \}          
    \end{aligned}
\end{equation}\par

常数项等于0的仿射函数称为线性布尔函数(linear Boolean functions),将全体n元线性布尔函数记作$L_n$。
$$
L_n = \{ f(x_1,x_2,\dots ,x_n) = \mathop{\oplus}\limits_{i=1}^n a_ix_i |a_i \in F_2 \}
$$
\end{definition}\par
可见,
$$
A_n = \{ L_n(x_1,x_2,\dots ,x_n) \oplus a_0 |a_i \in F_2 \ \}
$$
更多布尔函数的密码学性质将在第三章中具体介绍,下面将说明布尔函数与S盒的联系。
% %* =todo====不增加4)Walsh谱表示。
% 4)Walsh谱表示。
% \begin{definition}[布尔函数的Walsh变换]
%     假设布尔函数$f:\mathbb{F}_2^n \mapsto \mathbb{F}_2$在$w$的Walsh变换记作$S_f(w)$,有:
%     $$
%         S_f(w) = 
%     $$
% \end{definition}\par


%! 向量值布尔函数和S盒
\section{向量值布尔函数和S盒}
S盒的密码学性质和组成它的布尔函数之间密切相关。在考察一个S盒的密码学性质时,必须要把它当作一个整体看待,同时还要考虑到组成它的布尔函数之间的互相影响。
所以,\par
%! 加一个表格
在密码学算法中,S盒通常用一个直观的查询表格LUT(\emph{Look-Up Table}),来列出所有的输入值和对应的输出值,例如表格todo。
但是从密码分析的角度,使用更为清晰的数学语言来描述S盒是非常重要的,也有利于对密码学性质的研究和检测。一般地,我们使用向量值布尔函数(\emph{vectorial Boolean functions})来表示S盒。\par
$n\times m$的S盒本质上可以看作一个映射:$F(x) = (f_1(x), f_2(x),\dots , f_m(x)), x\in \mathbb{F}_2^n$,
相当于$m$个$n$元布尔函数的线性组合。数学上,可以把$n\times m$的S盒表示为一个$n$位输入,$m$位输出的向量值布尔函数:
$$
S:\mathbb{F}_2^n \mapsto \mathbb{F}_2^m
$$
所以也有一些文献将S盒直接称为向量值布尔函数。\par

与布尔函数类似,向量值布尔函数也可以由其正规代数型唯一表示。但不同的是,向量值布尔函数的ANF的系数在$\mathbb{F}_2^m$上,而不是$\mathbb{F}_2$。
\begin{definition}[向量值布尔函数的代数正规型]
    向量值布尔函数$S$:$\mathbb{F}_2^n \mapsto \mathbb{F}_2^n$可以如(\ref{def:Sanf})所示n元多项式形式表示且唯一,称为$S$的代数正规型(ANF, algebraic normal form)。
    \begin{equation}
        S(x_1,x_2,\dots ,x_n) = \mathop{\oplus}\limits_{u \in F_2}\alpha_u \prod\limits_{i=1}^n x_i^{u_i}\mbox{,其中}\alpha_u \in \mathbb{F}_2^m \label{def:Sanf}
    \end{equation}
\end{definition}\par

一般地,$n\times m$的S盒可以看作$m$个$n$元布尔函数的线性组合,其中有两种常见的布尔函数--坐标函数和分量函数。
在一些文献中,可能会将坐标函数和分量函数混为一谈。但严谨起见,两者需要区别开来,定义如下:\par
%! 坐标函数和分量函数,附录
\begin{definition}[S盒的坐标函数]
    $m$位输出的S盒$S$有$m$个坐标函数(\emph{coordinates}),分别表示$S$的第i位输出,记作$S_{e_i}$。易知坐标函数$S_{e_i}$
    是一个$n$元布尔函数:
    $$
    S_{e_i}:\mathbb{F}_2^n \mapsto \mathbb{F}
    $$
    其中,$\{e_i\}_{i<m}$表示$\mathbb{F}_2^n$上的标准偏差(\emph{standard basis})。
\end{definition}\par

\begin{definition}[S盒的分量函数]
    $n \times m$的S盒$S$有$2^m$个分量函数(\emph{components}),是$m$个坐标函数的线性组合,记作$S_{\lambda}$
    \begin{align*}
        S_{\lambda}& :\mathbb{F}_2^n \mapsto \mathbb{F}_2 \\
        & x \mapsto \lambda \cdot S(x),\lambda \in \mathbb{F}_2^m
    \end{align*}
\end{definition}\par
部分文献中,常见将$S_{e_i}$或$S_{\lambda}$统称为分量函数。但是检测在部分密码学性质过程中,例如
考察S盒是否为平衡时,两者必须区分。
\begin{definition}[S盒的平衡性]
    如果向量值函数$S:\mathbb{F}_2^n \mapsto \mathbb{F}_2^m$的非零分量函数,也即坐标函数的任意非零线性组合是平衡的,那么S具有平衡性。
\end{definition}
可以看出,平衡向量值函数的每一个坐标函数都是平衡的, 反过来, 每一个坐标函数都平衡, 并不意味着向量值函数是平衡函数,而是要求每一个非零分量函数平衡。

%! S盒的密码学性质

\chapter{S盒的密码学性质及其检测原理}
随着密码技术的发展,各种攻击方式层出不穷,于是S盒的安全性需要从多个方面进行考量,它抵御不同攻击的能力指标即为各种密码学性质。
在本章中,我们将按照S盒需要抵御的攻击方式来分类,对 S 盒重要的密码学性质进行介绍,给出检测性质的原理和作必要的解释说明。
%对S盒重要的密码学性质和检测原理进行介绍,最后给出检测函数的伪代码和作必要的解释说明。
\section{S盒密码学性质}
\subsection{抵御线性攻击的密码学性质}
\subsubsection{线性攻击}
在密码分析中,线性分析是最为常见的攻击方式之一。最早由Matsui\cite{matsui1993linear}
于1993年提出此概念,随后广泛适用于分组密码或者流密码的攻击中。线性攻击是一种已知明文攻击,目标是找到明文$P$、密文$C$和密钥$K$的若干比特的线性关系,称为线性逼近(\emph{linear approximation})。下面来考虑对于S盒的线性逼近。\par

% 假设$n\times m$的S盒的输入比特为$X=(x_1,\dots, x_n)$,输出比特为$Y=(y_1,\dots, y_m)$ 。其中易知$X$从$\{0,1\}^n$均匀独立地选取,即是指每一个坐标$x_i$定义了一个随机变量 $X_i$,取值于$\{0,1\}$,且偏差$\epsilon_i=0$,但显然输出比特$Y_i$不一定满足偏差为0。形如下式的线性表达式可以表示输入和输出比特之间存在的线性关系:
% $$
% X_{i_1} \oplus X_{i_2}\oplus \dots \oplus X_{i_u} \oplus Y_{j_1}\oplus Y_{j_1}\dots \oplus Y_{j_v}=0
% $$\par

% 在完全随机的情况下,上式成立的概率为1/2。如果上述表达式有很高或者很低的概率出现时,即与1/2相比偏移越多时,说明存在不随机性的缺陷,那么算法越容易受到线性攻击。

% 1todo或者:这样写\par

一般地,$n\times m$的S盒$S$的线性逼近形如下式,简称为($\alpha, \beta$),其中$x$表示输入比特,$S(x)$表示输出比特:
\begin{equation}\label{linearApp}
    \alpha \cdot x \oplus \beta \cdot S(x) = \alpha \cdot x \oplus S_{\beta}(x)
\end{equation}\par

在完全随机的情况下,上式等于0的概率为1/2。而概率偏离1/2越多,则说明不随机性越大,那么S盒越容易受到线性攻击。

\subsubsection{线性逼近表与非线性度}
% !\paragraph*{线性逼近表}
为了便于考察各个线性等式出现的概率,一般使用线性近似表(Linear Approximation Table,LAT),用于衡量S盒的各个输入的线性组合与输出的线性组合之间的距离。
首先定义偏移量(\emph{bias})表示上节中式(\ref{linearApp})为0成立的概率偏离1/2的程度,记作$\epsilon$。 对于$n\times m$的S盒$S$,其线性逼近($\alpha, \beta$)的bias可以这样表示:
$$
\epsilon_S(\alpha,\beta) = \lvert \frac{\#\{x|S_{\beta}(x)=\alpha\cdot x \}}{2^n} -1/2 \rvert
$$\par

% $S$的LAT表是一个$2^n \times 2^m$的表格,其中,LAT($\alpha, \beta$)是指第$\alpha$行、第$\beta$列的元素,值为$\epsilon_S(\alpha,\beta)$。\par
$S$的LAT表的建立是计算每一个$\epsilon_S(\alpha,\beta)$,并将其填入表的第$\alpha$行、第$\beta$列中,最后得到一个$2^n \times 2^m$的表格。
如果将表格的元素除以LAT的长度,可以得出对应的偏移概率。下面讨论如何衡量S盒的非线性度。\par

% !\paragraph*{非线性度}
S盒的非线性度最初由Pieprzyk等人在1988年提出\cite{pieprzyk1988towards},用于考察S盒的输入比特和输出比特之间的线性相关性。非线性度越高,抵抗线性分析攻击的能力越强\cite{冯登国2000分组密码的设计与分析}。
非线性度的计算可以通过S盒输入的线性组合与输出线性组合之间的汉明距离得到,可以采用计算线性逼近表的方式得到线性度。
\begin{equation}
    NL(S) = 2^{n-1}\: - \frac{1}{2} \; \mathop{max}\limits_{\alpha \in \mathbb{F}_2^m, \beta \in \mathbb{F}_2^n} \; |\epsilon(\alpha, \beta)|
\end{equation}\par

对于布尔函数$f$而言,其非线性度是$f$到全体仿射函数最小的汉明距离。两个布尔函数$f$和$g$之间的汉明距离等于值向量中值不同的位数,即$d_H (f,g)\triangleq wt(f \oplus g)$。
\begin{definition}[布尔函数的非线性度]
    对于布尔函数$f:\mathbb{F}_2^n \mapsto \mathbb{F}_2$,它的非线性$NL(f)$为:
    \begin{equation}
        NL(f) \triangleq \mathop{min} \limits_{g \in A(n)} \: d_H(f,g)= \mathop{min}\limits_{\alpha \in \mathbb{F}_2^n} \: |wt(f\oplus \varphi_{\alpha}) |
    \end{equation}\par
    其中,由式(\ref{affine_func})可知$A(n)$表示全体仿射函数。
\end{definition}\par

在分析布尔函数的线性性质或者其它密码学相关的性质时,Walsh变换是一个常用的工具。Walsh变换的定义如下:
\begin{definition}[布尔函数的Walsh变换]
    假设布尔函数$f:\mathbb{F}_2^n \mapsto \mathbb{F}_2$在$w$的Walsh变换记作$S_f(w)$,有:
    \begin{equation}\label{bfWalsh}
        W_f(w) = \mathop{\sum}\limits_{x \in \mathbb{F}_2^n} (-1)^{f(x)\oplus w\cdot x}
    \end{equation}\par
\end{definition}\par

从定义(\ref{bfWalsh})可以看出,$W_f(w)$描述的是一个布尔函数$f$与线性或者仿射函数的接近程度,实际上Walsh变换量化了布尔函数的线性程度。
有:
\begin{equation}
    \begin{aligned}
        L(f) &\triangleq \mathop{max}\limits_{\alpha \in \mathbb{F}_2^n}\: \lvert 2^n -2wt(f \oplus \varphi_{\alpha}) \rvert \:= \: \mathop{max}\limits_{\alpha \in \mathbb{F}_2^n} \lvert W_f(\varphi)\rvert \\
        NL(f) &= \frac{1}{2} - \mathop{max}\limits_{\alpha \in \mathbb{F}_2^n} \lvert W_f(\varphi)\rvert = 2^{n-1} - \frac{1}{2}L(f)
    \end{aligned}
\end{equation}

与$NL(f)$相似,可以给出S的线性度与非线性度的定义:
\begin{definition}[S盒的非线性度]
    向量值布尔函数$S:\mathbb{F}_2^n \mapsto \mathbb{F}_2^m$的非线性度是坐标函数的任意非零线性组合中的非线性度的最小值。
    \begin{equation}\label{Snonlinearity}
        NL(S) = \: \mathop{min}\limits_{\lambda \in \mathbb{F}_2^m \backslash \{0\}} NL(S_{\lambda}) = 2^{n-1} - \: \frac{1}{2}\mathop{max}\limits_{\alpha \in \mathbb{F}_2^n, \beta \in \mathbb{F}_2^m \backslash \{0\}} \lvert W_f(\alpha, \beta)\rvert
    \end{equation}
\end{definition}
由上可知,S盒非线性度的可以通过快速Walsh变换\cite{sosa2016calculating}的方法来计算,具体算法及伪代码见(todo)。


\subsection{抵御差分攻击的密码学性质}
\subsubsection{差分攻击}
差分密码分析\cite{DBLP:conf/crypto/BihamS90}是一种选择明文攻击,其基本思想是通过分析特定明文差分对相应密文差分的影响来获得可能性最大的密钥,它是目前对分组密码最为有效的攻击方法之一。要设计出一个安全的分组密码算法,必须保证给定任意非零的输入差分,没有高概率的输出差分。 一般来说,由于 S 盒是分组密码算法的非线性部分,如果一个S盒的固定差分对出现的概率越大,整个算法的固定差分对出现的概率也越大。因此,研究 S 盒的差分性质显得十分重要。
通常考察S盒抗差分密码分析的能力,主要与其差分均匀度和差分分布表有关\cite{冯登国2000分组密码的设计与分析}。

\subsubsection{差分分布表与差分均匀度}
%! \paragraph*{差分分布表}
与线性逼近表类似,$n\times m$的S盒的差分分布表\cite{DBLP:conf/fse/Nyberg94}(\emph{Differential Distribution Table, DDT})也定义为一个$2^n \times 2^m$的表格。通过 DDT 能够直观和快速地对 S 盒的差分性质进行了解和判断。下面对DDT的建立作简要说明。\par
对于向量值布尔函数$S:\mathbb{F}_2^n \mapsto \mathbb{F}_2^m$,在$a\in \mathbb{F}_2^n $方向的导数(\emph{derivative})定义为:
$$
    D_aS:\; x\; \mapsto \; S(x) \oplus S(x\oplus a),\; (x,a\in \mathbb{F}_2^n)
$$
将满足关系$S(x) \oplus S(x\oplus a) =b$的输入值$x$的数量记作$\delta_S(a,b)$:
\begin{equation}
    \delta_S(a,b)\triangleq \# \{x\in \mathbb{F}_2^n |\: S(x) \oplus S(x\oplus a)=b\} = | D_aS^{-1}(b)|
\end{equation}

对于所有的$(a,b)\in \mathbb{F}_2^n \times \mathbb{F}_2^m$对,将$\delta_S(a,b)$的值填入表的第$a$行、第$b$列中,得到的$2^n \times 2^m$的表格即为S盒的DDT表。
由于表格中的元素是对应差分对出现的次数。如果某些元素的值明显大于其他各元素的值,则这些元素的位置对差分攻击特别有用,为此引入差分均匀度的概念。

%! \paragraph*{差分均匀度}
\begin{definition}[S盒的差分均匀度]
    向量值布尔函数$S:\mathbb{F}_2^n \mapsto \mathbb{F}_2^m$的差分均匀度定义为其差分分布表中各元素的最大值,即:
    \begin{equation}
        DU(S) \triangleq \: \mathop{max}\limits_{a \in \mathbb{F}_2^n \backslash \{0\}, b\in \mathbb{F}_2^m } \delta_S(a,b)
    \end{equation}
\end{definition}

\subsubsection{鲁棒度}
在分组密码的迭代过程中,利用差分分布表DDT(S)的第一列中的非零元素个数这一特征攻击分组密码最为有效,因此除了要求差分均匀度$DU(S)$很小外,还应使DDT(S)的第一列包含尽可能少的非零元素,这一指标可用鲁棒度\cite{刘景伟2004aes}来衡量。
\begin{definition}[S盒的鲁棒度]
    给定$n \times m$的S盒$S$的差分均匀度为$DU(S)$,记$\sigma(S) =\mathop{\sum}\limits_{i=1}^{2^n-1}u(\lambda_{i0})$,其中$x=0$时$u(x)$等于0,否则为1。那么S盒的鲁棒性为:
\begin{equation}
    \eta(S) = (1-DU(S) \;/2^n)(1-\delta(S)\;/2^n)
\end{equation}\par
\end{definition}\par
S盒的鲁棒性越高,抵御利用迭代特征进行的密码攻击的性能越好。\par

\subsubsection{严格雪崩效应和扩散准则}
严格雪崩效应和扩散准则(Propagation)是用来衡量S盒的输出改变量对输入改变量的随机性。由上文可知,差分密码分析的本质取决于输入和输出改变量的不均匀分布,所以它们也是S盒重要的设计准则之一。\par
\begin{definition}[严格雪崩效应]
    $S(X)=(f_1 (X),\cdot,f_m (X)):\mathbb{F}_2^n \mapsto \mathbb{F}_2^m$满足严格雪崩准则是指改变输入的每一比特,每个输出比特改变的概率为1/2。
    换言之,对于任意非零向量$a\in \mathbb{F}_2^n$且$wt(a)=1$, $\delta_f(a,1) =2^{n-1}$。
\end{definition}
\begin{definition}[布尔函数的k次扩散准则]
    如果$n$元布尔函数$f:\mathbb{F}_2^n \mapsto \mathbb{F}_2$满足:对于非零向量$a \in \mathbb{F}_2^n$且$1\leq wt(a)\leq k$,$f(x)\oplus f(x\oplus a)$是平衡的,
    则称$f$关于非零向量$a \in \mathbb{F}_2^n$满足$k$次扩散准则。
\end{definition}

\begin{definition}[S盒的k次扩散准则]
    如果向量值布尔函数$S:\mathbb{F}_2^n \mapsto \mathbb{F}_2^m$的每个坐标函数均满足$k$次扩散准则,则称$S$满足$k$次扩散准则。    
\end{definition}
由上可知,满足一次扩散准则的函数即是满足严格雪崩准则的函数(SAC函数)。

\subsection{抵御回旋镖攻击的密码学性质}
\subsubsection{回旋镖攻击}
回旋镖攻击\cite{DBLP:conf/fse/Wagner99}(Boomerang attack),也称飞去来器攻击,是一种自适应选择明文和密文的攻击,是差分攻击的一个变体。
它的主要思想在于利用两个路径较短但概率较高的差分,来代替一个路径较长但概率较低的差分。
假设$E_0$和$E_1$是两个加密过程$E(\cdot)$的两个子过程,即$E=E_0 \circ E_1$,其中
$E_0$表示加密算法前半部分,且存在概率为$p$的差分路径$\alpha \to \beta$;
$E_1$表示加密算法后半部分,且存在概率为$q$的差分路径$\gamma \to \delta$。
如果两条差分路径相互独立,那么攻击者可以检测到一个概率为$p^2q^2$的回旋镖区分器,满足:
\begin{equation*}\label{boomerangPr}
    Pr[E^{-1}(E(P)\oplus \delta) \oplus E^{-1}(E(P\oplus \alpha) \oplus \delta) = \alpha] \;=\; p^2q^2
\end{equation*}
即攻击者能够以概率$p^2q^2$找到一个符合回旋镖区分器的明文对($P_1,P_2,P_3,P_4$),称为正确四元组。
而且满足$P_1\oplus P_2=\alpha, C_1\oplus C_2=\delta, C_3\oplus C_4= \delta,P_3\oplus P_4 = \alpha$。
对于加密算法$E$,正确四元组产生的概率为$p^2q^2$,所以当$pq>2^{-n}$时,回旋镖攻击成立。\par
2010年Dunkelman等人\cite{DBLP:conf/crypto/DunkelmanKS10}改进了回旋镖攻击,设计出一种 Sandwich 攻击:将分组密码算法$E$看成是三个子算法的级联,即:$E = E_0 \circ E_m \circ E_1$,并指出中间层的概率不是平方的\cite{DBLP:journals/tosc/BaoGLS19}。
Cid等人\cite{DBLP:conf/eurocrypt/CidHPSS18}针对$E_m$为单个S盒层的设计进行研究,并提出了一个类似于差分分布表的计算概率的的表,即为回旋镖连接表(\emph{Boomerang Connectivity Table, BCT})。下面来作简单说明。

\subsubsection{回旋镖连接表和回旋镖均匀度}
%! \paragraph*{回旋镖连接表}
    $n\times n$的可逆S盒$S$,令$a\in \mathbb{F}_2^n, b\in \mathbb{F}_2^n$分别为$E_0$输出差分和$E_1$的输入差分,有:
    $$
        \beta_S(a,b) \triangleq \# \{x\in \mathbb{F}_2^n | S^{-1}(S(x)\oplus b)\oplus S^{-1}(S(x\oplus a)\oplus b)\;=\; a \}
    $$
    类似地,将所有的$\beta_S(a,b)$值填入表的第$a$行、第$b$列中,得到的$2^n \times 2^n$的表格即为S盒的BCT表。  

%! \paragraph*{回旋镖均匀度}
\begin{definition}[回旋镖均匀度]
    $n\times n$的可逆S盒$S$的差分均匀度定义为其差分分布表中各元素的最大值,记作$BU(S)$:
    \begin{equation}
        BU(S) = \mathop{max}\limits_{a,b \in \mathbb{F}_2^n \backslash \{ 0\} } \beta_S(a,b)
    \end{equation}
\end{definition}\par
Dunkelman\cite{DBLP:journals/iacr/Dunkelman18}提出了一个快速构建BCT表的方法,将在本章第三节中具体介绍。

\subsection{抵御代数攻击的密码学性质}
通常我们使用S盒的代数次数来衡量S盒的代数非线性程度,代数次数的大小一定程度上反映了S盒的线性复杂度。代数次数越大,S盒的线性复杂度越高,越难用线性表达式逼近\cite{刘景伟2003rijndael},其抵抗代数攻击和插值攻击的能力越强。
对于基于S盒的密码算法来说,S盒常常提供了整个算法的代数次数的上界。
S盒的代数次数有两种不同方式的定义,一种称为代数次数,另一种称为一致代数次数。\par
\subsubsection{代数次数}
由(\ref{bfAlgebraicDegree})可知,布尔函数的代数次数为:
$deg(f) = max\{wt(u)|u\in \mathbb{F}_2^n,\alpha_u \ne 0 \in \mathbb{F}_2^m \enspace in \enspace ANF_f \}$ 。
那么根据布尔函数推广到向量值函数,可以得到其代数次数计算如下:
\begin{definition}[S盒的代数次数]
    向量值布尔函数$S:\mathbb{F}_2^n \mapsto \mathbb{F}_2^m$的代数次数记作$deg(S)$,有:
    \begin{equation}
        deg(S) \triangleq max\{wt(u)|u\in \mathbb{F}_2^n,\alpha_u \ne 0 \in \mathbb{F}_2^m \enspace in \enspace ANF_S \}
    \end{equation}
\end{definition}\par
为了抵抗插值攻击和代数攻击,密码体系中使用的布尔函数应当具有高的代数次数,因此,向量值函数的各坐标函数也应当具有高的代数次数。
但在实际应用中,往往还进一步要求各坐标函数的任意非零线性组合的代数次数也比较高, 于是引入一致代数次数的概念:
%\cite{密码函数的安全性指标分析}。
% \begin{definition}[S盒一致代数次数]
    
% \end{definition}
\begin{equation}
    min\; deg(S) \triangleq \; \mathop{min}\limits_{\lambda \in \mathbb{F}_2^m \backslash \{0\}}\; deg(S_{\lambda})
\end{equation}
其中,$S_{\lambda}$为S盒的非零分量函数,即坐标函数的任意非零线性组合,而不是坐标函数$S_{e_i}$。



\section{检测原理及函数实现}
本小节列出了主要的检测函数的实现及其原理。在函数实现过程中,我们尽可能地搜集并选取相对先进,时间复杂度和空间复杂度都更低的创新算法进行计算。
\subsection{非线性度计算}
在计算非线性度时,我们没有采用传统的线性逼近算法,而是采用基于Walsh-Hadamard变换的计算发誓,只需要经过两次for循环求解S盒坐标函数的非零线性组合的真值表,再对列表进行遍历就能求解出最终S盒的非线性度。因此,其时间复杂度与空间复杂度均为$O(n×2^n)$。相比于基于线性逼近表的非线性度算法,该算法的时间复杂度与空间复杂度为原来的$1/m$。
流程图及其伪代码如下,图片测试图片插入测试:\par
% \includegraphics{nonlinearityLC.png}
% \includegraphics{nonlinearityW.png}
\begin{figure}
    \centering
    \includegraphics[scale=0.5]{nonlinearityLC.png}
    \hspace{1in}
    \includegraphics[scale=0.5]{nonlinearityW.png}
    \caption{test}
  \end{figure}

\subsection{差分均匀性计算}
差分均匀度的计算可以通过遍历差分分布表中的最大值得到。

\chapter{测试}
\section{算法环境测试}
\begin{algorithm}
    \caption{Evaluating the Mobius Transform}
    \begin{algorithmic}
        \REQUIRE{$a[i], 0\leq i < 2^n$}
        \ENSURE{$b = M(a)$}
        \STATE $x\gets0$
        \IF{$x\leq0$}
        \STATE{$y\gets-1$}
        \ELSE
        \STATE{$y\gets1$}
        \ENDIF
    \end{algorithmic}
\end{algorithm}


\printbibliography
\backmatter
\begin{appendixes}
    \chapter{这是一个附录}
\end{appendixes}
\end{document}

%?
%*
%todo
%!

%--定义
% \begin{definition}[]
% \end{definition}\par
%--Fn
% \mathbb{F}_2^n \mapsto \mathbb{F}_2^n
%--
%todo
% 在引言处性质增加引用 鲁棒性 \cite{刘景伟2004aes}
% 查一下差分propagation 怎么翻译
% 引用标注在哪